\documentclass[12pt, letterpaper, twoside]{article}
\usepackage[utf8]{inputenc}
\usepackage{graphicx}
\usepackage{amsfonts}
\usepackage[margin=2cm,nohead]{geometry}
\graphicspath{ {images/} }
\begin{document}
\title{IN5520 Mandatory 2}
\author{Espen Lønes}
\date{\today}
\maketitle
\ \\
Task 1)\\
Plots of the glcms of the 4 textures for each direction:\\
\includegraphics[scale=0.4]{"task1_d1"}\\
\includegraphics[scale=0.4]{"task1_d2"}\\
\includegraphics[scale=0.4]{"task1_d3"}\\
\includegraphics[scale=0.4]{"task1_d4"}\\
\ \\
I don't think one direction will be enough to discriminate between all the textures. Because for all the directions at least two of the textures have very similar glcms.\\
Therefore i will chose two directions. As for which directions, i think dx=1,dy=0 and dx=0,dy=-1 will give a good chance for differentiating the textures.\\
Because each pair of textures are different in at least one direction.\\
(for convenience i will from now on call dx=1,dy=0 for d1. And dx=0,dy=-1 for d2)\\
e.g. texture 1 and 4 are different in both d1 and d2. But texture 1 and 2 are different in d1 but quite similar in d2.\\
Also texture 2 and 3 are quite different from themselves when comparing d1 and d2.\\
These differences between d1 and d2 should give us a lot of good information to give the classifier.\\
\ \\
\ \\
Task 2)\\
\ \\
Based on the glcms produced by d1 and d2. We see that texture 1 and 2 has most of their energy split between Q1 and Q4, for d1 and d2.\\
While texture 3 and 4 has most of theirs in just Q1, again for d1 and d2.\\
Furthermore, texture 1 has some energy in Q2 and Q3 for both d1 and d2.\\
As for texture 2, is also has some energy in Q2 and Q3 for d2. But for d1 it has almost no energy in Q2 and Q3.\\
A similar relation is also true for textures 3 and 4.\\
Texture 3 has some energy in Q2, Q3 and Q4 (still most in Q1) for d1. And also a tiny amount for d2 as well. Texture 4 on the other hand appears to have no energy at all for Q2, Q3 and Q4 for both d1 and q2.\\
\ \\
Therefore, using Q1 and Q4 for d1 and d2 (+4 quadrants total), we can separate textrure 1 and 2 from tesxture 3 and 4.\\
Then we may use Q2 and Q3 for d1 (+2 quadrant total) to separate texture 1 from 2.\\
We may then use Q2, Q3 and Q4 for d1, and posibly Q4 for d2\\
(+0 quadrants total as all are already used) to separate texture 3 from 4.\\
\ \\
We therefore need 6 quadrants to discriminate between all 4 textures.\\
\ \\
\ \\
Task 3)\\
\ \\
Se 'task3.m' and 'computequadrants.m' for how glcms and quadrants are computed, as well as plotting of the feature images.\\
\includegraphics[scale=0.5]{"task3_quadrants.png"}\\
\end{document}